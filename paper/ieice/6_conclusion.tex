\section{結論}
本論文では、内視鏡画像からの簡易カルテ生成システムとしての深層学習モデルの学習を行った。既存研究では食道がんのみの予測かつ手動でのすべての画像へのラベル付けが必要となっていたのに対して、本研究では医者が実際に作成した簡易カルテから学習データとなるマルチラベルを自動で生成した。さらに食道がんのみに関わらず、頻出病名15個を含むマルチラベルを学習させることに成功した。既存研究では再現率に最適したモデルになっていたのに対し、本研究では適合率と再現率がともに高くなるようにF1-Scoreに最適化されたモデルとなっている。そのため病変の見逃しと誤検知の両方が少ないモデルであるといえる。また推論時には、モデルで出力したマルチラベルを二値化する際のしきい値を変えた出力を複数予測として提供することにより、信頼度の異なる段階的な予測を可能にした。これにより医師の診断の際の支援として大きく役立つモデルになったといえる。今後の展望としては2つ挙げられる。一つ目はデータセットの複雑化である。データセット作成の際には病名ラベルだけでなく階層化した付加情報もマルチラベルかできるようにしたが、学習が煩雑で進まなくなったために病名ラベルのみを使用した。階層化した付加情報も学習が可能になるとより詳細な簡易カルテの生成が可能になる。二つ目は学習モデルの改良である。今回は個別に画像とマルチラベルを学習させる手法と患者ごとに画像をまとめる手法を提案したが、画像特徴を個別にとりつつ系列データを扱うモデルで患者ごとに特徴をまとめてマルチラベルと学習させることが可能になれば、より精度の高い予測が可能になると考えられる。
