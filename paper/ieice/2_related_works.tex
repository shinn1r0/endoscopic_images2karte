\section{関連研究}
深層学習モデルに内視鏡画像を学習させた研究として、胃部分の内視鏡画像から胃癌の有無を判定したもの\cite{stomach_cancer}や食道部分の内視鏡画像から食道癌の有無を判定したもの\cite{stomach_cancer}がある。これらの研究では画像を一枚ずつラベル付けし、これらをConvolutional Neural Network (CNN)\cite{CNN}を用いて学習した。用いたモデルはSingle Shot Multibox Detector (SSD)\cite{SSD}であり、これは16層のCNNで物体検出に特化したものである。またどちらの研究も一つの病変に特化したものである。食道癌の検知の研究では384人の患者から得られた8428枚の画像を訓練データとし、47人の食道がん患者と50人の食道がんではない患者から得られた1118枚の画像をテストデータとした。この研究では食道がんの早期発見のために10mm以下の小さな病変を検知できるモデルの学習に特化した。その結果95\%という高い再現率を記録したが、適合率は40\%ほどとなっている。またこの研究では通常の内視鏡画像であるwhite-light image (WLI)と内視鏡機器に搭載されている狭帯域光観察によって病変を見やすくした画像であるnarrow-band imaging (NBI)をそれぞれ分けたものと包括的に扱ったものの3つの場合で比較している。この比較ではどの場合でも再現率は高いが適合率が低くなっている。また胃癌の検知の研究も同様に適合率が低く精度があまり良くない。

これらの研究の課題としては大きく3つのことが挙げられる。一つ目は各画像への手動でのラベリングである。これにより学習データの準備に非常に時間がかかる上に、データ拡張に対応できない。これらのモデルは診断の補助には利用できるが医師の作業効率化の支援にはなっていない。二つ目は一つ目とも関連するが、学習データが少ないことである。現在深層学習モデルの学習には数十万の画像を用いるのが一般的であるが、これらの研究では1万に満たないものであり、結果の適合率が非常に低いのはこの課題によるものでもあると考えられる。三つ目はSSDというモデルの選択である。SSDは16層という比較的浅いCNNモデルであるが、物体検出というタスクに特化することで高い精度を記録したものである。機械学習における物体検出とは景色のなかにある動物の検出といったタスクであり、内視鏡画像の中にあるわずかな病変を検出するような高難度なタスクとは異なっている。この研究での結果はこのモデルの選択に大きく影響を受けてしまっていると考えられる。このため用いるCNNのモデルはより深い特徴抽出ができる多層のモデルかつ、大きさの異なる複数の特徴の組み合わせを学習できる残差構造をもっているものが望ましいと考えられる。
本論文においては以上の3つの課題を解決することを目標とする。
