% コンパイル方法 
% pdfpLaTeX->pBibTex->pdfpLaTeX

%%「論文」,「レター」,「レター(C分冊)」,「技術研究報告」などのテンプレート
%% v3.2 [2019/03/19]
%% 1. 「論文」
\documentclass[paper]{ieicej}
% \documentclass[technicalreport]{ieicej}%研究会フォーマット使用%自分で追加
\bibliographystyle{junsrt} %reference.bibを読み込むのに必要%自分で追加
\usepackage{amssymb}%自分で追加
\usepackage[dvipdfmx]{graphicx}%自分で追加
\usepackage{subfig}%自分で追加
\usepackage{comment}%自分で追加
\def\BibTeX{{\rmfamily B\kern-.05em{\scshape i\kern-.025em b}\kern-.08em
 T\kern-.1667em\lower.7ex\hbox{E}\kern-.125em X}}%自分で追加
\usepackage{amsmath}
\usepackage{amssymb}
\usepackage{bm}               % ベクトル表記(ボールド体)
\newcommand{\argmax}{\mathop{\rm arg~max}\limits}
\newcommand{\argmin}{\mathop{\rm arg~min}\limits}


%\documentclass[invited]{ieicej}% 招待論文
%\documentclass[survey]{ieicej}% サーベイ論文
%\documentclass[comment]{ieicej}% 解説論文
%\usepackage[dvips]{graphicx}
%\usepackage[dvipdfmx]{graphicx,xcolor}
\usepackage[T1]{fontenc}
\usepackage{lmodern}
\usepackage{textcomp}
\usepackage{latexsym}
\usepackage{url}
%\usepackage[fleqn]{amsmath}
%\usepackage{amssymb}

\setcounter{page}{1}

\field{}
\jtitle{内視鏡画像からの簡易カルテ自動生成システム}
\etitle{An Automatic Creation Support System for \\Draft of Medical Chart from Endoscopic Images}
\authorlist{%
 \authorentry{}{}{}\MembershipNumber{}
 %\authorentry{和文著者名}{英文著者名}{所属ラベル}\MembershipNumber{}
 %\authorentry[メールアドレス]{和文著者名}{英文著者名}{所属ラベル}\MembershipNumber{}
 %\authorentry{和文著者名}{英文著者名}{所属ラベル}[現在の所属ラベル]\MembershipNumber{}
 \authorentry[naito@soft.ics.keio.ac.jp]{内藤 慎一郎}{Shinichiro Naito}{Keio}\MembershipNumber{}
 \authorentry[hagiwara@keiojp]{萩原 将文}{Masafumi Hagiwara}{Keio}\MembershipNumber{}
}
\affiliate[Keio]{慶應義塾大学大学院理工学研究科, 〒223--8522 神奈川県横浜市港北区日吉3-14-1}{Graduate School of Science and Technology, Keio University, Hiyoshi 3--14--1, Kouhoku-ku, Kanagawa, 223--8522 Japan}
%\affiliate[所属ラベル]{和文所属}{英文所属}
%\paffiliate[]{}
%\paffiliate[現在の所属ラベル]{和文所属}

\begin{document}
\begin{abstract}
%和文あらまし 500字以内
本論文では深層学習モデルを用いて内視鏡画像からの簡易カルテ作成を自動化する医師の支援システムを提案する。
患者の検査画像には、病変部がある場合でも多くの病変部のない画像が含まれている。
従来の研究では画像単位での学習と予測を行っているため、患者単位での予測が極めて困難であった。
本研究では、内視鏡検査における複数の画像を患者単位に学習と予測を行えるようなシステムを作成した。
このシステムは簡易カルテの情報を患者単位にまとめたデータセットと、患者単位の画像とデータセットを学習できるモデルからなる。
また従来の研究では一種類の病変のみの予測が可能なシステムであった。
本研究では患者に病変が存在するかの判定や複数の病変情報をまとめたマルチラベルデータを作成した。
これらの改善によって、患者単位での複数の病変の予測が可能なシステムとなった。
さらに既存研究では全ての画像に手動でラベルデータを付与していたのに対し、本研究では、ラベルデータを医師が実際に作成した簡易カルテから自動で生成した。
この学習データを用いて二次元と三次元の畳み込みニューラルネットワークをそれぞれ複数モデルずつ訓練し比較した。
そしていくつかのモデルで、患者に病変が存在するかの判定でF1-Scoreが98\%以上、複数の病変情報の予測でF1-Scoreが54\%以上と、高い推論性能を記録した。
また推論時の予測ラベルの生成において、複数の信頼度を持つ予測を生成した。
複数の病変情報の予測において、信頼度の高い予測では適合率が77\%以上、信頼度が低い予測では再現率85\%以上を記録した。
これによって本研究におけるシステムは、医師の診断時の判断において、信頼度の異なる複数の予測を利用することが可能なものとなった。

\end{abstract}
\begin{keyword}
%和文キーワード 4〜5語
医療分野、作業自動化、画像認識、異常検知、マルチラベル学習
\end{keyword}
\begin{eabstract}
%英文アブストラクト 100 words
In this paper, we devised an automatic creation support system for a draft of medical charts from endoscopic images using a deep learning model.
In previous studies, learning and prediction were performed on an image basis, and prediction on a patient basis was impossible.
It was a system that could predict only one type of lesion.
In this study, multi-label teacher data consisting of a label indicating whether one or more lesions exist and a plurality of lesion labels was used.
This made it possible to predict multiple lesions on a patient basis.
In addition, in the existing study, label data was manually assigned to all images, whereas in this study, multi-label data was automatically generated from a simple chart actually created by a doctor.
Using this learning data, we trained and compared several models of 2D and 3D convolutional neural networks. And some models recorded high inference performance.
We also implemented a function that supports doctors' judgment by generating predictions with multiple reliability levels when generating prediction labels.


\end{eabstract}
\begin{ekeyword}
%英文キーワード
medical domain, automatic creation, image recognition, anomaly detection, multi label learning
\end{ekeyword}
\maketitle

\section{まえがき}
現在機械学習の分野において深層学習による画像処理、系列データ処理が高い精度を記録している。このため文字や画像や音声などの認識や生成、翻訳や検索といった自然言語処理に幅広く用いられるようになっている。また化合物の性質予測やゲノム構造の解析、株価の予測のように、生物学や化学や経済学などの情報工学分野を超えた様々な分野においても用いられてきている。

この流れの中で医学分野においても様々な用途で深層学習が用いられてきている。例として以下のようなものが挙げられる。
(例3つくらい)

医学分野において深層学習を用いた自動化によって診断の精度向上や作業の効率化が期待できる用途として、内視鏡画像からの簡易カルテの生成が挙げられる。現在簡易カルテの生成は人手によって行われており、各患者ごとに5分から10分ほどの時間がかかっている。さらに大量の2重チェック負担によって専門医が疲弊するなどといった問題も発生している。
これらの作業が自動化されることによる医師の負担削減が大きく期待できる。
また病変見落としが医師によっては2割以上といったデータもあり、深層学習を用いた精度の高い予測が可能になり病変の見落としが減ることによって患者に対する大きな寄与も期待できる。

しかしながら深層学習を医学分野に用いることに関していくつかの問題がある。一つ目は倫理的な問題である。医療における診断という大きな責任を伴う判断において、深層学習のような機械学習モデルを用いることには責任の所在がなくなるという課題がある。現在の深層学習モデルはブラックボックスであり、判断の根拠が不明となっている。このため医療分野における責任ある判断をまかせることにはまだ至っていない。二つ目はプライバシーの問題である。患者の診断データや内視鏡画像は重要度の非常に高いプライバシーデータであり、実際の推論システムにおいて患者のデータが特定可能な形で出力されてしまうことを必ず避けなければならない。

本研究では簡易カルテという精密検査や診断の確定がなされる前の作業における支援システムとして導入するため、最終的な判断の責任は医師にある。また学習データの作成では、簡易カルテから学習データのマルチラベルを作成するに当たって、質的診断の項目を自動で収集してマルチラベル化するため、プライバシー情報が完全に学習データに含まれない。学習においても、内視鏡画像とマルチラベルデータのみを深層学習モデルに通した。

このように本論文で取り上げる用途での深層学習モデルの利用は利点が多い上に、現在深層学習を医療分野に取り入れる際に課題となることにも対処できていると言える。

既存研究では各患者に対して数十枚ある内視鏡画像すべてに手動で病名情報をラベリングしており、深層学習に必要な数十万の学習データを用意するために膨大な時間がかかっている。全ての画像においてラベリングをした教師あり学習であるために非常に高い精度を記録しているが、推論時も各画像単位で行われるため、画像を用いた診断として用いることはできるが簡易カルテの作成の自動化には至っていない。このため本研究では医師が作成した簡易カルテに対して自然言語処理を行い、教師データとしてマルチラベルを作成した。このマルチラベルデータは二つの部分から構成されている。一番目のラベルは内視鏡画像の中に病変があるか否かを示す値が格納されており、二番目以降は頻出病名に対応した複数のラベルが並んでいる。マルチラベルデータは簡易カルテの病名情報を要約したものとして用いることができる。内視鏡画像とマルチラベルデータの関係性を直接学習させることによって、学習データの作成と学習全ての工程を自動化することに成功した。

学習手法に関しては、大きく分けて二つの手法で実験した。一つ目の手法は各患者における数十枚の画像をそれぞれ一枚ずつ入力し、出力をマルチラベルから損失を計算し、モデルを学習するものである。二つ目の手法では各患者の全ての画像を三次元データとして用いてモデルを学習した。

本論文では作成した学習データを二つの手法においてそれぞれ複数の深層学習モデルで学習と推論を行い、結果を比較した。さらにそれぞれのモデルの結果の傾向から、今後のシステムの推論性能向上のための改善案も提案する。

以下、第2章で関連研究について述べ、第3章で提案手法について、第4章で評価実験、第5章で考察、第6章で結論を述べる。

深層学習モデルに内視鏡画像を学習させた研究として、胃部分の内視鏡画像から胃癌の有無を判定したもの\cite{stomach_cancer}や食道部分の内視鏡画像から食道癌の有無を判定したもの\cite{stomach_cancer}がある。
これらの研究では画像を一枚ずつラベル付けし、これらをConvolutional Neural Network (CNN)\cite{CNN}を用いて学習した。
用いたモデルはSingle Shot Multibox Detector (SSD)\cite{SSD}であり、これは16層のCNNで物体検出に特化したものである。
またどちらの研究も一つの病変に特化したものである。
食道癌の検知の研究では384人の患者から得られた8,428枚の画像を訓練データとし、47人の食道がん患者と50人の食道がんではない患者から得られた1,118枚の画像をテストデータとした。
この研究では食道がんの早期発見のために10mm以下の小さな病変を検知できるモデルの学習に特化した。
その結果95\%という高い再現率を記録したが、適合率は40\%ほどとなっている。
またこの研究では通常の内視鏡画像であるwhite-light image (WLI)と内視鏡機器に搭載されている狭帯域光観察によって病変を見やすくした画像であるnarrow-band imaging (NBI)をそれぞれ分けたものと包括的に扱ったものの3つの場合で比較している。
この比較ではどの場合でも再現率は高いが適合率が低くなっている。
また胃癌の検知の研究も同様に適合率が低く精度があまり良くない。

これらの研究の課題としては大きく3つのことが挙げられる。
一つ目は各画像への手動でのラベリングである。
これにより学習データの準備に非常に時間がかかる上に、データ拡張に対応できない。
これらのモデルは診断の補助には利用できるが医師の作業効率化の支援にはなっていない。
二つ目は一つ目とも関連するが、学習データが少ないことである。
現在深層学習モデルの学習には数十万の画像を用いるのが一般的であるが、これらの研究では1万に満たないものであり、結果の適合率が非常に低いのはこの課題によるものでもあると考えられる。
三つ目はSSDというモデルの選択である。
SSDは16層という比較的浅いCNNモデルであるが、物体検出というタスクに特化することで高い精度を記録したものである。
機械学習における物体検出とは景色のなかにある動物の検出といったタスクであり、内視鏡画像の中にあるわずかな病変を検出するような高難度なタスクとは異なっている。
この研究での結果はこのモデルの選択に大きく影響を受けてしまっていると考えられる。
このため用いるCNNのモデルはより深い特徴抽出ができる多層のモデルかつ、大きさの異なる複数の特徴の組み合わせを学習できる残差構造をもっているものが望ましいと考えられる。
本論文においては以上の3つの課題を解決することを目標とする。


\section{内視鏡画像からの簡易カルテ自動生成システム}
提案手法をデータセットの作成、モデルの作成、モデルの学習、評価指標から説明する。
\subsection{データセットの作成}
\subsection{モデルの作成}
\subsection{モデルの学習}
\subsection{評価指標}

\section{実験条件の設定}
\subsection{モデルの損失関数}
マルチラベル予測の学習のためバイナリ交差エントロピーを用いた。
\subsection{モデルの最適化手法}
確率的勾配降下法\cite{SGD}を拡張したAdaptive Moment Estimation (Adam)\cite{Adam}を使用した。
これは勾配の大きさと更新量によって学習率を変化させていく方法で、様々なタスクで高性能を記録している。
\subsection{モデルの学習回数とバッチサイズ}
モデルの学習回数はすべて25にした。
これは訓練時における損失関数の推移から判断した。
またバッチサイズはDenseNet121で50、DenseNet161で24、3D-ResNetで3とした。
これは使用した計算機のメモリの容量によって決めた。
\section{評価指標}
評価指標は正解率、完全一致正解率、再現率、適合率、F1-Scoreの5つとした。
以下本文中の各データとは、画像を個別に入力する手法では各画像、画像を患者ごとにまとめて入力する手法では画像を時間軸で連結した3次元データを示す。
\subsection{正解率 (Acc)}
\begin{equation}
各データにおける正解率 = \frac{正解したラベルの数}{マルチラベルにおける全てのラベルの数}
\end{equation}
を計算し、これをすべてのデータにおいて平均を計算した。
\subsection{完全一致正解率 (AllAcc)}
\begin{equation}
AllAcc=\frac{マルチラベルにおける全てのラベルで一致したデータの数}{全てのデータの数}
\end{equation}
\subsection{適合率 (Precision)}
\begin{equation}
Precision = \frac{真陽性}{真陽性+偽陽性}
\end{equation}

全てのデータにおける適合率を計算するために、各データにおける混合行列を集計し、その後適合率を計算した。
\subsection{再現率 (Recall)}
\begin{equation}
Recall = \frac{真陽性}{真陽性+偽陰性}
\end{equation}

全てのデータにおける再現率を計算するために、各データにおける混合行列を集計し、その後再現率を計算した。
\subsection{F1-Score}
適合率と再現率がトレードオフの関係であるため、2つの指標を総合的に判断するためにF1-Scoreを用いた。

\begin{equation}
F1{\rm \mathchar"712D}Score = \frac{Precision * Recall}{(Precision + Recall) / 2}
\end{equation}

\subsection{評価指標とモデルの性能の関係性}
\begin{itemize}
    \item 正解率\\
        正解率は指標としてモデルの性能をあまり評価できない。
        マルチラベル予測の際にこの指標を用いると、マルチラベルのクラス数が多いほど真陰性の割合が多くなり、実際の予測がほとんど行われていなくても高い数値が出るからである。
    \item 完全一致正解率\\
        完全一致正解率は指標としてモデルの性能をあまり評価できない。
        マルチラベル予測の際にこの指標を用いると、マルチラベルのクラス数が多いほど全てを一致させることが困難になり、ほぼ全てのクラスで正解しているもとと全く正解していないものを区別できない。
    \item 適合率\\
        適合率はマルチラベル予測の指標として一般的に用いられる。
        適合率は陽性であると予測したものの中で、実際に陽性であるものの割合である。
        これは本用途においては病変があると予測したものの中で、実際に病変があったものの割合となっており、誤検知の少なさの指標と言える。
    \item 再現率\\
        再現率はマルチラベル予測の指標として一般的に用いられる。
        再現率は実際に陽性であるものの中で、陽性であると予測できたものの割合である。
        これは本用途においては病変があるデータの中で、病変があると予測できたものの割合となっており、見落としの少なさの指標と言える。
    \item F1-Score\\
        F1-Scoreはモデルの性能を最も表していると言える。
        本実験ではこの数値が高いものを良いモデルとして評価する。
\end{itemize}

\section{DenseNetを用いた内視鏡画像からの\\マルチラベル予測}
\subsection{最も性能の良いモデルの選定}
\label{sec:ex1}
\subsubsection{実験概要}
\ref{sec:densenet}で述べたように、DenseNetをチューニングした4つのモデルを提案手法とし、実験で用いた。
層の数の異なる2つのDenseNetであるDenseNet121とDenseNet161の2パターンのモデル
において、分類器部分の構造を拡張した分類器拡張ありと分類器拡張なしの2パターンの構造を用いた。
分類器拡張は、分類器拡張がないモデルが一層の全結合層からなるのに対して、分類器拡張ありのモデルでは二つの全結合層とその層の間にバッチ正規化と活性化関数とドロップアウトを適用したものである。
それらの合計4つのモデルを以下のように定義する。

\begin{enumerate}
    \item DenseNet121-分類器拡張なし (モデル1)
    \item DenseNet121-分類器拡張あり (モデル2)
    \item DenseNet161-分類器拡張なし (モデル3)
    \item DenseNet161-分類器拡張あり (モデル4)
\end{enumerate}
以上の4モデルで実験を行い、性能を比較した。
実験の流れは以下のように行った。
各画像をマルチラベルと関連付けしたデータセットを用いた。
画像をモデルに入力し出力されたマルチラベルと教師データのマルチラベルから損失を計算し最適化を行った。
実験の出力結果は、学習過程での損失とF1-Scoreの推移と、検証データを用いた際の各評価指標の値となる。

\subsubsection{実験結果}
モデル1の結果を図\ref{fig:densenet121_result_process}と表\ref{tb:densenet121}に示す。
図\ref{fig:densenet121_result_process}は学習過程での損失とF1-Scoreの推移を示している。
表\ref{tb:densenet121}は検証データを用いた際の予測の結果を示す。
この表のTotalは図\ref{fig:multilabel}における全てのラベルにおける結果を、LesionとLabelは図\ref{fig:multilabel}のラベルの1番目と2番目以降に分けて計算した結果を示している。

\begin{figure}[htbp]
    \begin{center}
        \includegraphics[width=150mm]{./fig/densenet121process.png}
        \caption{モデル1の学習過程}
        \label{fig:densenet121_result_process}
    \end{center}
\end{figure}

\begin{table}[tb]
    \caption[]{モデル1の検証結果}
    \label{tb:densenet121}
    \centering
    \normalsize
    \begin{tabular}{c|c|r} \hline
        Total & Acc (\%) & 92.4 \\ \cline{2-3}
         & AllAcc (\%) & 32.7 \\ \cline{2-3}
         & F1-Score (\%) & 74.0 \\ \cline{2-3}
         & Precision (\%) & 88.3 \\ \cline{2-3}
         & Recall (\%) & 63.6 \\ \hline
        Lesion & Acc (\%) & 97.1 \\ \cline{2-3}
         & F1-Score (\%) & 98.5 \\ \cline{2-3}
         & Precision (\%) & 98.2 \\ \cline{2-3}
         & Recall (\%) & 98.8 \\ \hline
        Label & Acc (\%) & 92.1 \\ \cline{2-3}
         & AllAcc (\%) & 33.9 \\ \cline{2-3}
         & F1-Score (\%) & 54.5 \\ \cline{2-3}
         & Precision (\%) & 77.2 \\ \cline{2-3}
         & Recall (\%) & 42.1 \\ \hline
    \end{tabular}
\end{table}

\newpage
モデル2の結果を図\ref{fig:densenet121_e_result_process}と表\ref{tb:densenet121_e}に示す。
図\ref{fig:densenet121_e_result_process}は学習過程での損失とF1-Scoreの推移を示している。
表\ref{tb:densenet121_e}は検証データを用いた際の予測の結果を示す。
この表のTotalは図\ref{fig:multilabel}における全てのラベルにおける結果を、LesionとLabelは図\ref{fig:multilabel}のラベルの1番目と2番目以降に分けて計算した結果を示している。

\begin{figure}[htbp]
    \begin{center}
        \includegraphics[width=150mm]{./fig/densenet121_e_p02process.png}
        \caption{モデル2の学習過程}
        \label{fig:densenet121_e_result_process}
    \end{center}
\end{figure}

\begin{table}[tb]
    \caption[]{モデル2の検証結果}
    \label{tb:densenet121_e}
    \centering
    \normalsize
    \begin{tabular}{c|c|r} \hline
        Total & Acc (\%) & 92.4 \\ \cline{2-3}
         & AllAcc (\%) & 32.8 \\ \cline{2-3}
         & F1-Score (\%) & 73.9 \\ \cline{2-3}
         & Precision (\%) & 88.8 \\ \cline{2-3}
         & Recall (\%) & 63.2 \\ \hline
        Lesion & Acc (\%) & 97.4 \\ \cline{2-3}
         & F1-Score (\%) & 98.7 \\ \cline{2-3}
         & Precision (\%) & 98.0 \\ \cline{2-3}
         & Recall (\%) & 99.4 \\ \hline
        Label & Acc (\%) & 92.1 \\ \cline{2-3}
         & AllAcc (\%) & 34.1 \\ \cline{2-3}
         & F1-Score (\%) & 53.9 \\ \cline{2-3}
         & Precision (\%) & 78.1 \\ \cline{2-3}
         & Recall (\%) & 41.2 \\ \hline
    \end{tabular}
\end{table}

\newpage
モデル3の結果を図\ref{fig:densenet161_result_process}と表\ref{tb:densenet161}に示す。
図\ref{fig:densenet161_result_process}は学習過程での損失とF1-Scoreの推移を示している。
表\ref{tb:densenet161}は検証データを用いた際の予測の結果を示す。
この表のTotalは図\ref{fig:multilabel}における全てのラベルにおける結果を、LesionとLabelは図\ref{fig:multilabel}のラベルの1番目と2番目以降に分けて計算した結果を示している。

\begin{figure}[htbp]
    \begin{center}
        \includegraphics[width=150mm]{./fig/densenet161process.png}
        \caption{モデル3の学習過程}
        \label{fig:densenet161_result_process}
    \end{center}
\end{figure}

\begin{table}[tb]
    \caption[]{モデル3の検証結果}
    \label{tb:densenet161}
    \centering
    \normalsize
    \begin{tabular}{c|c|r} \hline
        Total & Acc (\%) & 92.4 \\ \cline{2-3}
         & AllAcc (\%) & 32.3 \\ \cline{2-3}
         & F1-Score (\%) & 73.9 \\ \cline{2-3}
         & Precision (\%) & 87.8 \\ \cline{2-3}
         & Recall (\%) & 63.8 \\ \hline
        Lesion & Acc (\%) & 97.3 \\ \cline{2-3}
         & F1-Score (\%) & 98.6 \\ \cline{2-3}
         & Precision (\%) & 97.8 \\ \cline{2-3}
         & Recall (\%) & 99.4 \\ \hline
        Label & Acc (\%) & 92.0 \\ \cline{2-3}
         & AllAcc (\%) & 33.7 \\ \cline{2-3}
         & F1-Score (\%) & 54.3 \\ \cline{2-3}
         & Precision (\%) & 76.6 \\ \cline{2-3}
         & Recall (\%) & 42.1 \\ \hline
    \end{tabular}
\end{table}

\newpage
モデル4の結果を図\ref{fig:densenet161_e_result_process}と表\ref{tb:densenet161_e}に示す。
図\ref{fig:densenet161_e_result_process}は学習過程での損失とF1-Scoreの推移を示している。
表\ref{tb:densenet161_e}は検証データを用いた際の予測の結果を示す。
この表のTotalは図\ref{fig:multilabel}における全てのラベルにおける結果を、LesionとLabelは図\ref{fig:multilabel}のラベルの1番目と2番目以降に分けて計算した結果を示している。

\begin{figure}[htbp]
    \begin{center}
        \includegraphics[width=150mm]{./fig/densenet161_eprocess.png}
        \caption{モデル4の学習過程}
        \label{fig:densenet161_e_result_process}
    \end{center}
\end{figure}

\begin{table}[tb]
    \caption[]{モデル4の検証結果}
    \label{tb:densenet161_e}
    \centering
    \normalsize
    \begin{tabular}{c|c|r} \hline
        Total & Acc (\%) & 92.5 \\ \cline{2-3}
         & AllAcc (\%) & 33.4 \\ \cline{2-3}
         & F1-Score (\%) & 74.0 \\ \cline{2-3}
         & Precision (\%) & 88.6 \\ \cline{2-3}
         & Recall (\%) & 63.6 \\ \hline
        Lesion & Acc (\%) & 98.0 \\ \cline{2-3}
         & F1-Score (\%) & 98.9 \\ \cline{2-3}
         & Precision (\%) & 98.4 \\ \cline{2-3}
         & Recall (\%) & 99.5 \\ \hline
        Label & Acc (\%) & 92.1 \\ \cline{2-3}
         & AllAcc (\%) & 34.3 \\ \cline{2-3}
         & F1-Score (\%) & 54.2 \\ \cline{2-3}
         & Precision (\%) & 77.3 \\ \cline{2-3}
         & Recall (\%) & 41.7 \\ \hline
    \end{tabular}
\end{table}

\newpage
\subsection{しきい値を変化させた際の適合率と再現率の変化の確認}
\label{sec:ex11}
\subsubsection{実験概要}
\ref{sec:ex1}の実験で、マルチラベル全体でのF1-Scoreが最も高かったモデル1を本実験で用いる。
本実験では、モデルが出力したマルチラベルの各ラベルを二値化する際のしきい値を変化させた。
しきい値は0.1から0.9まで0.1刻みで変化させ、その値における適合率、再現率、F1-Scoreを出力した。

図\ref{fig:densenet121_result_threshold}はテスト推論において、モデルが出力したマルチラベルの値を二値化する際のしきい値を変化させた際の、適合率、再現率、F1-Scoreの変化を示している。

\begin{figure}[htbp]
    \begin{center}
        \includegraphics[width=150mm]{./fig/densenet121threshold.png}
        \caption{しきい値を変化させた際の適合率と再現率の変化}
        \label{fig:densenet121_result_threshold}
    \end{center}
\end{figure}

\newpage
\section{3D-ResNetを用いた内視鏡画像からのマルチラベル予測}
\subsection{最も性能の良いモデルの選定}
\label{sec:ex2}
\subsubsection{実験概要}
\ref{sec:resnet3d}で述べたように、3D-ResNetをチューニングした4つのモデルを提案手法とし、実験で用いた。
3D-ResNetの特徴抽出部分の最終層で平均プーリングを用いたものと最大プーリングを用いたものの2つのモデルにおいて、分類器部分の構造を拡張した分類器拡張ありと分類器拡張なしの2パターンの構造を用いた。
分類器拡張は、分類器拡張がないモデルが一層の全結合層からなるのに対して、分類器拡張ありのモデルでは二つの全結合層とその層の間にバッチ正規化と活性化関数とドロップアウトを適用したものである。
それらの合計4つのモデルを以下のように定義する。

\begin{enumerate}
    \item 3D-ResNet-AveragePool-分類器拡張なし (モデル1)
    \item 3D-ResNet-AveragePool-分類器拡張あり (モデル2)
    \item 3D-ResNet-MaxPool-分類器拡張なし (モデル3)
    \item 3D-ResNet-MaxPool-分類器拡張あり (モデル4)
\end{enumerate}
以上の4モデルで実験を行い、性能を比較した。
実験の流れは以下のように行った。
各患者ごとに画像を時間軸で連結した三次元データとマルチラベルと関連付けしたデータセットを用いた。
三次元データをモデルに入力し出力されたマルチラベルと教師データのマルチラベルから損失を計算し最適化を行った。
実験の出力結果は、学習過程での損失とF1-Scoreの推移と、検証データを用いた際の各評価指標の値となる。

\subsubsection{実験結果}
モデル1の結果を図\ref{fig:resnet3d_result_process}と表\ref{tb:resnet3d}に示す。
図\ref{fig:resnet3d_result_process}は学習過程での損失とF1-Scoreの推移を示している。
表\ref{tb:resnet3d}は検証データを用いた際の予測の結果を示す。
この表のTotalは図\ref{fig:multilabel}における全てのラベルにおける結果を、LesionとLabelは図\ref{fig:multilabel}のラベルの1番目と2番目以降に分けて計算した結果を示している。

\begin{figure}[htbp]
    \begin{center}
        \includegraphics[width=150mm]{./fig/resnet3dprocess.png}
        \caption{モデル1の学習過程}
        \label{fig:resnet3d_result_process}
    \end{center}
\end{figure}

\begin{table}[tb]
    \caption[]{モデル1の検証結果}
    \label{tb:resnet3d}
    \centering
    \normalsize
    \begin{tabular}{c|c|r} \hline
        Total & Acc (\%) & 92.2 \\ \cline{2-3}
         & AllAcc (\%) & 28.5 \\ \cline{2-3}
         & F1-Score (\%) & 72.9 \\ \cline{2-3}
         & Precision (\%) & 86.5 \\ \cline{2-3}
         & Recall (\%) & 63.0 \\ \hline
        Lesion & Acc (\%) & 98.8 \\ \cline{2-3}
         & F1-Score (\%) & 99.4 \\ \cline{2-3}
         & Precision (\%) & 98.9 \\ \cline{2-3}
         & Recall (\%) & 99.9 \\ \hline
        Label & Acc (\%) & 91.7 \\ \cline{2-3}
         & AllAcc (\%) & 28.9 \\ \cline{2-3}
         & F1-Score (\%) & 50.8 \\ \cline{2-3}
         & Precision (\%) & 71.8 \\ \cline{2-3}
         & Recall (\%) & 39.4 \\ \hline
    \end{tabular}
\end{table}

\newpage
モデル2の結果を図\ref{fig:resnet3d_e_result_process}と表\ref{tb:resnet3d_e}に示す。
図\ref{fig:resnet3d_e_result_process}は学習過程での損失とF1-Scoreの推移を示している。
表\ref{tb:resnet3d_e}は検証データを用いた際の予測の結果を示す。
この表のTotalは図\ref{fig:multilabel}における全てのラベルにおける結果を、LesionとLabelは図\ref{fig:multilabel}のラベルの1番目と2番目以降に分けて計算した結果を示している。

\begin{figure}[htbp]
    \begin{center}
        \includegraphics[width=150mm]{./fig/resnet3d_eprocess.png}
        \caption{モデル2の学習過程}
        \label{fig:resnet3d_e_result_process}
    \end{center}
\end{figure}

\begin{table}[tb]
    \caption[]{モデル2の検証結果}
    \label{tb:resnet3d_e}
    \centering
    \normalsize
    \begin{tabular}{c|c|r} \hline
        Total & Acc (\%) & 91.7 \\ \cline{2-3}
         & AllAcc (\%) & 23.3 \\ \cline{2-3}
         & F1-Score (\%) & 71.3 \\ \cline{2-3}
         & Precision (\%) & 85.7 \\ \cline{2-3}
         & Recall (\%) & 61.1 \\ \hline
        Lesion & Acc (\%) & 98.8 \\ \cline{2-3}
         & F1-Score (\%) & 99.4 \\ \cline{2-3}
         & Precision (\%) & 99.0 \\ \cline{2-3}
         & Recall (\%) & 99.8 \\ \hline
        Label & Acc (\%) & 91.2 \\ \cline{2-3}
         & AllAcc (\%) & 23.5 \\ \cline{2-3}
         & F1-Score (\%) & 48.3 \\ \cline{2-3}
         & Precision (\%) & 69.9 \\ \cline{2-3}
         & Recall (\%) & 36.9 \\ \hline
    \end{tabular}
\end{table}

\newpage
モデル3の結果を図\ref{fig:resnet3d_m_result_process}と表\ref{tb:resnet3d_m}に示す。
図\ref{fig:resnet3d_m_result_process}は学習過程での損失とF1-Scoreの推移を示している。
表\ref{tb:resnet3d_m}は検証データを用いた際の予測の結果を示す。
この表のTotalは図\ref{fig:multilabel}における全てのラベルにおける結果を、LesionとLabelは図\ref{fig:multilabel}のラベルの1番目と2番目以降に分けて計算した結果を示している。

\begin{figure}[htbp]
    \begin{center}
        \includegraphics[width=150mm]{./fig/resnet3d_mprocess.png}
        \caption{モデル3の学習過程}
        \label{fig:resnet3d_m_result_process}
    \end{center}
\end{figure}

\begin{table}[tb]
    \caption[]{モデル3の検証結果}
    \label{tb:resnet3d_m}
    \centering
    \normalsize
    \begin{tabular}{c|c|r} \hline
        Total & Acc (\%) & 92.0 \\ \cline{2-3}
         & AllAcc (\%) & 26.3 \\ \cline{2-3}
         & F1-Score (\%) & 71.8 \\ \cline{2-3}
         & Precision (\%) & 88.7 \\ \cline{2-3}
         & Recall (\%) & 60.4 \\ \hline
        Lesion & Acc (\%) & 99.0 \\ \cline{2-3}
         & F1-Score (\%) & 99.5 \\ \cline{2-3}
         & Precision (\%) & 99.0 \\ \cline{2-3}
         & Recall (\%) & 99.9 \\ \hline
        Label & Acc (\%) & 91.5 \\ \cline{2-3}
         & AllAcc (\%) & 27.0 \\ \cline{2-3}
         & F1-Score (\%) & 48.4 \\ \cline{2-3}
         & Precision (\%) & 75.0 \\ \cline{2-3}
         & Recall (\%) & 35.7 \\ \hline
    \end{tabular}
\end{table}

\newpage
モデル4の結果を図\ref{fig:resnet3d_e_m_result_process}と表\ref{tb:resnet3d_e_m}に示す。
図\ref{fig:resnet3d_e_m_result_process}は学習過程での損失とF1-Scoreの推移を示している。
表\ref{tb:resnet3d_e_m}は検証データを用いた際の予測の結果を示す。
この表のTotalは図\ref{fig:multilabel}における全てのラベルにおける結果を、LesionとLabelは図\ref{fig:multilabel}のラベルの1番目と2番目以降に分けて計算した結果を示している。

\begin{figure}[htbp]
    \begin{center}
        \includegraphics[width=150mm]{./fig/resnet3d_e_mprocess.png}
        \caption{モデル4の学習過程}
        \label{fig:resnet3d_e_m_result_process}
    \end{center}
\end{figure}

\begin{table}[tb]
    \caption[]{モデル4の検証結果}
    \label{tb:resnet3d_e_m}
    \centering
    \normalsize
    \begin{tabular}{c|c|r} \hline
        Total & Acc (\%) & 91.8 \\ \cline{2-3}
         & AllAcc (\%) & 26.5 \\ \cline{2-3}
         & F1-Score (\%) & 71.6 \\ \cline{2-3}
         & Precision (\%) & 87.0 \\ \cline{2-3}
         & Recall (\%) & 60.9 \\ \hline
        Lesion & Acc (\%) & 98.2 \\ \cline{2-3}
         & F1-Score (\%) & 99.1 \\ \cline{2-3}
         & Precision (\%) & 98.9 \\ \cline{2-3}
         & Recall (\%) & 99.3 \\ \hline
        Label & Acc (\%) & 91.4 \\ \cline{2-3}
         & AllAcc (\%) & 26.9 \\ \cline{2-3}
         & F1-Score (\%) & 48.7 \\ \cline{2-3}
         & Precision (\%) & 72.2 \\ \cline{2-3}
         & Recall (\%) & 36.8 \\ \hline
    \end{tabular}
\end{table}

\newpage
\subsection{しきい値を変化させた際の適合率と再現率の変化の確認}
\label{sec:ex22}
\subsubsection{実験概要}
\ref{sec:ex2}の実験で、マルチラベル全体でのF1-Scoreが最も高かったモデル1を本実験で用いる。
本実験では、モデルが出力したマルチラベルの各ラベルを二値化する際のしきい値を変化させた。
しきい値は0.1から0.9まで0.1刻みで変化させ、その値における適合率、再現率、F1-Scoreを出力した。

図\ref{fig:resnet3d_result_threshold}はテスト推論において、モデルが出力したマルチラベルの値を二値化する際のしきい値を変化させた際の、適合率、再現率、F1-Scoreの変化を示している。

\begin{figure}[htbp]
    \begin{center}
        \includegraphics[width=150mm]{./fig/resnet3dthreshold.png}
        \caption{しきい値を変化させた際の適合率と再現率の変化}
        \label{fig:resnet3d_result_threshold}
    \end{center}
\end{figure}

\section{考察}
\subsection{各実験における考察}
\subsubsection{DenseNetを用いた内視鏡画像からの\\マルチラベル予測における考察}
\subsubsection{3D-ResNetを用いた内視鏡画像からの\\マルチラベル予測における考察}
\subsection{全体における考察}

\section{結論}
結論

\bibliography{./reference/reference}


% \ack %% 謝辞

%\bibliographystyle{sieicej}
%\bibliography{myrefs}
% \begin{thebibliography}{99}% 文献数が10未満の時 {9}
% \bibitem{}
% \end{thebibliography}

% \appendix
% \section{}

\begin{biography}
% \profile{}{}{}
%\profile{会員種別}{名前}{紹介文}% 顔写真あり
%\profile*{会員種別}{名前}{紹介文}% 顔写真なし
\profile*{n}{内藤 慎一郎}{2018慶大理工情報卒.現在同大大学院修士課程在学中.ニューラルネットワークに関する研究に従事.}
\profile*{m}{萩原 将文}{1982年慶大・工・電気卒.1987年同大学院博士課程修了.工博.同年同大助手.現在,同大教授.1991-92年度スタンフォード大学訪問研究員.視覚・言語・感性情報処理とその融合の研究に従事.1990年IEEE Consumer Electronics Society論文賞,1996年日本ファジィ学会著述賞,2004年,2014年日本感性工学会論文賞,2013年日本神経回路学会最優秀研究賞,2018年日本知能情報ファジィ学会論文賞受賞.}
\end{biography}

\end{document}



%% 2. 「レター」
\documentclass[letter]{ieicej}
%\usepackage[dvips]{graphicx}
%\usepackage[dvipdfmx]{graphicx,xcolor}
\usepackage[T1]{fontenc}
\usepackage{lmodern}
\usepackage{textcomp}
\usepackage{latexsym}
%\usepackage[fleqn]{amsmath}
%\usepackage{amssymb}

\setcounter{page}{1}

\typeofletter{研究速報}
%\typeofletter{紙上討論}
%\typeofletter{問題提起}
%\typeofletter{ショートノート}
\field{}
\jtitle{}
\etitle{}
\authorlist{%
 \authorentry{}{}{}{}\MembershipNumber{}
 %\authorentry{和文著者名}{英文著者名}{会員種別}{所属ラベル}\MembershipNumber{}
 %\authorentry{和文著者名}{英文著者名}{会員種別}{所属ラベル}[現在の所属ラベル]\MembershipNumber{}
}
\affiliate[]{}{}
%\affiliate[所属ラベル]{和文所属}{英文所属}
%\paffiliate[]{}
%\paffiliate[現在の所属ラベル]{和文所属}

\begin{document}
\maketitle
\begin{abstract}
%和文あらまし 120字以内
\end{abstract}
\begin{keyword}
%和文キーワード 4〜5語
\end{keyword}
\begin{eabstract}
%英文アブストラクト 50 words
\end{eabstract}
\begin{ekeyword}
%英文キーワード
\end{ekeyword}

% \section{まえがき}


% \ack %% 謝辞

%\bibliographystyle{sieicej}
%\bibliography{myrefs}
% \begin{thebibliography}{99}% 文献数が10未満の時 {9}
% \bibitem{}
% \end{thebibliography}

% \appendix
% \section{}

\end{document}


%% 3. 「レター(C分冊)」
\documentclass[electronicsletter]{ieicej}
%\usepackage[dvips]{graphicx}
%\usepackage[dvipdfmx]{graphicx,xcolor}
\usepackage[T1]{fontenc}
\usepackage{lmodern}
\usepackage{textcomp}
\usepackage{latexsym}
%\usepackage[fleqn]{amsmath}
%\usepackage{amssymb}

\setcounter{page}{1}

\field{}
\jtitle{}
\etitle{}
\authorlist{%
 \authorentry{}{}{}{}\MembershipNumber{}
 %\authorentry{和文著者名}{英文著者名}{会員種別}{所属ラベル}\MembershipNumber{}
 %\authorentry{和文著者名}{英文著者名}{会員種別}{所属ラベル}[現在の所属ラベル]\MembershipNumber{}
}
\affiliate[]{}{}
%\affiliate[所属ラベル]{和文所属}{英文所属}
%\paffiliate[]{}
%\paffiliate[現在の所属ラベル]{和文所属}

\begin{document}
\begin{abstract}
%和文あらまし 120字以内
\end{abstract}
\begin{keyword}
%和文キーワード 4〜5語
\end{keyword}
\begin{eabstract}
%英文アブストラクト 50 words
\end{eabstract}
\begin{ekeyword}
%英文キーワード
\end{ekeyword}
\maketitle

\section{まえがき}


\ack %% 謝辞

%\bibliographystyle{sieicej}
%\bibliography{myrefs}
\begin{thebibliography}{99}% 文献数が 10 未満の時 {9}
\bibitem{}
\end{thebibliography}

\appendix
\section{}

\end{document}



%% 4. 「技術研究報告」
\documentclass[technicalreport]{ieicej}
%\usepackage[dvips]{graphicx}
%\usepackage[dvipdfmx]{graphicx,xcolor}
\usepackage[T1]{fontenc}
\usepackage{lmodern}
\usepackage{textcomp}
\usepackage{latexsym}
%\usepackage[fleqn]{amsmath}
%\usepackage{amssymb}

\jtitle{}
\jsubtitle{}
\etitle{}
\esubtitle{}
\authorlist{%
 \authorentry[]{}{}{}
% \authorentry[メールアドレス]{和文著者名}{英文著者名}{所属ラベル}
}
\affiliate[]{}{}
%\affiliate[所属ラベル]{和文勤務先\\ 連絡先住所}{英文勤務先\\ 英文連絡先住所}

\begin{document}
\begin{jabstract}
%和文あらまし
\end{jabstract}
\begin{jkeyword}
%和文キーワード
\end{jkeyword}
\begin{eabstract}
%英文アブストラクト
\end{eabstract}
\begin{ekeyword}
%英文キーワード
\end{ekeyword}
\maketitle

\section{はじめに}


%\bibliographystyle{sieicej}
%\bibliography{myrefs}
\begin{thebibliography}{99}% 文献数が10未満の時 {9}
\bibitem{}
\end{thebibliography}

\end{document}
