本論文では深層学習モデルを用いて内視鏡画像からの簡易カルテ作成を自動化する医師の支援システムを提案する。
従来の研究では図\ref{fig:overview}のように画像単位での学習と予測を行っており、患者単位での予測が不可能であった。
また一種類の病変のみの予測が可能なシステムであった。
本研究では図\ref{fig:overview}のように、病変が一つ以上存在するかのラベルと複数の病変ラベルから構成されるマルチラベルの教師データを用いた。
これによって患者単位での複数の病変の予測が可能なシステムとなった。
また既存研究では全ての画像に手動でラベルデータを付与していたのに対し、本研究では、マルチラベルデータを医師が実際に作成した簡易カルテから自動で生成した。
この学習データを用いて二次元と三次元の畳み込みニューラルネットワークをそれぞれ複数モデルずつ訓練し比較した。そしていくつかのモデルで高い推論性能を記録した。
また推論時の予測ラベルの生成において、複数の信頼度を持つ予測を生成することで、医師の判断の支援を行う機能も実装した。
