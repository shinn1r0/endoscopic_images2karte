\section{内視鏡画像からの簡易カルテ自動生成システム}
提案手法をデータセットの作成、モデルの作成、モデルの学習、評価指標から説明する。
すべての実装コードは筆者のgithubのレポジトリ (https://github.com/shinn1r0\\/endoscopic\_images2karte)にある。またプライバシー情報のためデータセット作成の元となる内視鏡画像と簡易カルテのデータ、それらから生成したデータセットは提供しない。
\subsection{データセットの作成}
本研究では画像を個別にマルチラベルと関連付けで学習する手法と各患者ごとの全ての画像をまとめてマルチラベルと関連付けで学習する手法の二つの手法で実験を行ったため、データセットも二つ用意した。簡易カルテからのマルチラベル生成と各画像における処理は共通であるが、その後の処理はそれぞれの手法で異なっている。
\subsubsection{簡易カルテからのマルチラベル生成}
簡易カルテからのマルチラベル生成は以下の手順で行った。
\begin{enumerate}
    \item すべて異常なしの患者を抽出
    \item 簡易カルテ内の質的診断を言語処理
    \item 処理後の内容を全ての患者でまとめてカテゴリー化
    \item 頻出病名カテゴリー選択
    \item 各患者ごとにマルチラベル生成
    \item マルチラベルのone-hotベクトル化
\end{enumerate}
詳細を述べる。 (1)では簡易カルテ内の全ての行で異常なしとなっている患者を抜き出し、 (2)から (4)の処理には通していない。 (2)では簡易カルテ内の質的診断の項目のみを取り出し、言語処理を行った。この言語処理では文書であったり単語や数値の羅列であったりする質的診断をまず単語の系列データへと変換した。次にこれらを形態素解析や事前に用意した病名リストを用いた検索などを通して病名を表す単語とそれらに付随する情報を示す単語を抽出した。その際に複雑な付随情報を持つような病名は固有の処理を加えている。これにより簡易カルテの質的診断の項目では、各行が、先頭に病名単語があり、その後に付加情報単語が続く形式になった。 (3)では (2)の処理後の内容を全ての患者を通して集計し、同じ単語を同じラベル付けをして辞書に格納した。 (4)では辞書内で出現回数を計測し、指定した回数以上のラベルを頻出病名ラベルとして決定した。 (5) (6)では各患者ごとにone-hot化したマルチラベルを生成した。マルチラベルは1番目のラベルには異常が一つでもある患者は1を、それ以外の患者は0が格納されている。これには (1)で抽出した情報を用いた。マルチラベルの2番目以降のラベルには頻出病名ラベルがある。 (2)から (4)で作成した情報を元に、各患者の質的診断の項目に頻出病名が存在する場合は1を、存在しない場合は0が格納されている。
\subsubsection{各画像における処理}
すべての画像は$256 \times 256$ピクセルにサイズ変更した。その後画像のRGBチャンネルそれぞれを正規化した。正規化のパラメータはRが平均0.485、標準偏差0.229、Gが平均0.456、標準偏差0.224、Bが平均0.406、標準偏差0.225に設定した。これらの値はすべて深層学習フレームワークのPytorchで推奨されている値を用いた。
\subsubsection{データセット分割}
\begin{itemize}
    \item 画像を個別に入力する手法
    \item 画像を患者ごとにまとめて入力する手法
\end{itemize}
\subsection{モデルの作成}
\begin{itemize}
    \item 画像を個別に入力する手法
    \item 画像を患者ごとにまとめて入力する手法
\end{itemize}
\subsection{モデルの学習}
\begin{itemize}
    \item 画像を個別に入力する手法
    \item 画像を患者ごとにまとめて入力する手法
\end{itemize}
\subsection{評価指標}
