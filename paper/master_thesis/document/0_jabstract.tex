本論文では深層学習モデルを用いて内視鏡画像からの簡易カルテ作成を自動化する医師の支援システムを提案する。
従来の研究では画像単位での学習と予測を行っており、患者単位での予測が不可能であった。
本研究では、内視鏡検査における複数の画像を患者単位に学習と予測を行えるように、データセットとモデルを作成した。
また従来の研究では一種類の病変のみの予測が可能なシステムであった。
本研究では患者に病変が存在するかの判定や複数の病変情報をまとめたラベルを作成した。
これらの改善によって、患者単位での複数の病変の予測が可能なシステムとなった。
さらに既存研究では全ての画像に手動でラベルデータを付与していたのに対し、本研究では、ラベルデータを医師が実際に作成した簡易カルテから自動で生成した。
この学習データを用いて二次元と三次元の畳み込みニューラルネットワークをそれぞれ複数モデルずつ訓練し比較した。そしていくつかのモデルで、患者に病変が存在するかの判定でF1-Scoreが98\%以上、複数の病変情報の予測でF1-Scoreが50\%以上と、高い推論性能を記録した。
また推論時の予測ラベルの生成において、複数の信頼度を持つ予測を生成した。
複数の病変情報の予測において、信頼度の高い予測では適合率が87\%以上、信頼度が低い予測では再現率81\%以上を記録した。
これによって本研究におけるシステムは医師の判断の支援を行うものとしても利用できるようになった。
