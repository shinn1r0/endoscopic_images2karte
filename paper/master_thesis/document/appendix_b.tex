簡易カルテからマルチラベルを生成する際に行った言語処理の詳細を述べる。
\section{言語処理の流れ}
\begin{enumerate}
    \item カルテを各行ごとに取り出す。
    \item 入力された文をすべて半角に変換する。
    \item 以下の記号や特定の単語で文を単語列に分割する。
    \begin{itemize}
        \item 句点
        \item 読点
        \item 先頭にある数字
        \item カンマ
        \item ピリオド
        \item 中点
        \item コロン
        \item セミコロン
        \item スペース
        \item 全角スペース
        \item タブ (\textbackslash t)
        \item 改行 (\textbackslash n, \textbackslash r, \textbackslash rn)
        \item に対して
        \item による
        \item にて
    \end{itemize}
    \item 単語列の中で単語なしの要素があれば削除する。
    \item 頻出する表記ゆれに対応する。
    \begin{itemize}
        \item 疑う所見なし、示唆する所見なし、所見なしを所見なしに統一
        \item CRTX、CRTx、CRTx後をCRTx後に統一
        \item 疑い、疑うを疑いに統一
        \item ELPS後、ELPS後瘢痕をELPS後瘢痕に統一
    \end{itemize}
    \item 度合いや付加状態を取り出す。
    \begin{itemize}
        \item 軽度
        \item 重度
        \item 疑い
        \item 術後
    \end{itemize}
    \item 大きさを取り出す。
    \begin{itemize}
        \item 正規表現 ([0-9]* \textbackslash .*[0-9]*[a-z]*m[a-z]*) で取り出す。
    \end{itemize}
    \item 特定の病気を優先的に取り出す。
    \begin{itemize}
        \item 食道癌
        \item マロリーワイス症候群
        \item ヘルニア
        \item ポリープ
        \item カンジダ症
        \item ヨード不染
        \item 咽頭喉頭炎
        \item 十二指腸炎
        \item 静脈拡張
        \item 潰瘍
        \item リンパ腫
        \item ペンタサ
    \end{itemize}
    \item 特定の病気を付加情報とともに取り出す。
    \begin{itemize}
        \item 慢性胃炎
        \begin{itemize}
            \item RAC
            \item 胃底腺ポリープ
            \item 稜線状発赤
            \item 扁平隆起
            \item ヘマチン付着
            \item 地図状発赤
            \item 色調逆転現症
        \end{itemize}
        \item 逆流性食道炎
        \begin{itemize}
            \item grade
            \item LA
        \end{itemize}
        \item 胃術後
        \begin{itemize}
            \item どの部分を摘出したか(全部・上半分・下半分)
            \item どのような手法で接続したか
        \end{itemize}
    \end{itemize}
    \item 残りの要素を付加情報として追加する。
\end{enumerate}
すべての要素追加時に以下の品詞を含む要素を除外した。
\begin{itemize}
    \item 助詞
    \item 接続詞
    \item 接頭辞
    \item 助動詞
    \item 副詞可能名詞
\end{itemize}
